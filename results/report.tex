\documentclass{article}
\usepackage{graphicx}
\usepackage{longtable}

\begin{document}

\title{Optimization Results Report}
\author{Your Name}
\date{\today}
\maketitle

\section{Introduction}
This report contains the results of the optimization process, including the final parameter estimates and likelihood values for various models and specifications.

\section{Optimization Results}

\subsection{Parameter Estimates}
Below is the table showing the optimized parameters for different models and specifications.

\begin{longtable}{|l|l|l|l|l|l|l|l|l|l|l|l|}
\hline
\textbf{Parameter} & \textbf{RBEKK\_Scalar} & \textbf{RBEKK\_Diagonal} & \textbf{RBEKK\_CP} & \textbf{OGARCH\_Scalar} & \textbf{OGARCH\_Diagonal} & \textbf{OGARCH\_CP} & \textbf{GOGARCH\_Scalar} & \textbf{GOGARCH\_Diagonal} & \textbf{GOGARCH\_CP} & \textbf{RDCC\_Scalar} & \textbf{RDCC\_Diagonal} & \textbf{RDCC\_CP} \\
\hline
\endfirsthead

\hline
\textbf{Parameter} & \textbf{RBEKK\_Scalar} & \textbf{RBEKK\_Diagonal} & \textbf{RBEKK\_CP} & \textbf{OGARCH\_Scalar} & \textbf{OGARCH\_Diagonal} & \textbf{OGARCH\_CP} & \textbf{GOGARCH\_Scalar} & \textbf{GOGARCH\_Diagonal} & \textbf{GOGARCH\_CP} & \textbf{RDCC\_Scalar} & \textbf{RDCC\_Diagonal} & \textbf{RDCC\_CP} \\
\hline
\endhead

\hline
\endfoot

\input{thetaD_table.tex}

\end{longtable}

\subsection{Likelihood Values}
Below are the likelihood values for each model and specification.

% Add likelihood values here, you can use a similar table structure or any other format you prefer.

\end{document}

